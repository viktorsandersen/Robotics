%%%%%%%%%%%%%%%%%%%%%%%%%%%%%%%%%%%%%%%%%
% Beamer Presentation
% LaTeX Template
% Version 1.0 (10/11/12)
%
% This template has been downloaded from:
% http://www.LaTeXTemplates.com
%
% License:
% CC BY-NC-SA 3.0 (http://creativecommons.org/licenses/by-nc-sa/3.0/)
%
%%%%%%%%%%%%%%%%%%%%%%%%%%%%%%%%%%%%%%%%%

%----------------------------------------------------------------------------------------
%	PACKAGES AND THEMES
%----------------------------------------------------------------------------------------

\documentclass{beamer}

\mode<presentation> {

% The Beamer class comes with a number of default slide themes
% which change the colors and layouts of slides. Below this is a list
% of all the themes, uncomment each in turn to see what they look like.

\usetheme{default}
%\usetheme{AnnArbor}
%\usetheme{Antibes}
%\usetheme{Bergen}
%\usetheme{Berkeley}
%\usetheme{Berlin}
%\usetheme{Boadilla}
%\usetheme{CambridgeUS}
%\usetheme{Copenhagen}
%\usetheme{Darmstadt}
%\usetheme{Dresden}
%\usetheme{Frankfurt}
%\usetheme{Goettingen}
%\usetheme{Hannover}
%\usetheme{Ilmenau}
%\usetheme{JuanLesPins}
%\usetheme{Luebeck}
%\usetheme{Madrid}
%\usetheme{Malmoe}
%\usetheme{Marburg}
%\usetheme{Montpellier}
%\usetheme{PaloAlto}
%\usetheme{Pittsburgh}
%\usetheme{Rochester}
%\usetheme{Singapore}
%\usetheme{Szeged}
%\usetheme{Warsaw}

% As well as themes, the Beamer class has a number of color themes
% for any slide theme. Uncomment each of these in turn to see how it
% changes the colors of your current slide theme.

%\usecolortheme{albatross}
%\usecolortheme{beaver}
%\usecolortheme{beetle}
%\usecolortheme{crane}
%\usecolortheme{dolphin}
%\usecolortheme{dove}
%\usecolortheme{fly}
%\usecolortheme{lily}
%\usecolortheme{orchid}
%\usecolortheme{rose}
%\usecolortheme{seagull}
%\usecolortheme{seahorse}
%\usecolortheme{whale}
%\usecolortheme{wolverine}

%\setbeamertemplate{footline} % To remove the footer line in all slides uncomment this line
%\setbeamertemplate{footline}[page number] % To replace the footer line in all slides with a simple slide count uncomment this line

%\setbeamertemplate{navigation symbols}{} % To remove the navigation symbols from the bottom of all slides uncomment this line
}

\usepackage{graphicx} % Allows including images
\usepackage{booktabs} % Allows the use of \toprule, \midrule and \bottomrule in tables
\usepackage{url}
\usepackage[T1]{fontenc}
\usepackage[utf8]{inputenc}
\usepackage{listings}
\usepackage{adjustbox}
\lstset{language=C++,
  basicstyle=\footnotesize\ttfamily,
  keywordstyle=\footnotesize\color{blue}\ttfamily,
}
%----------------------------------------------------------------------------------------
%	TITLE PAGE
%----------------------------------------------------------------------------------------

\title[Lab 8]{Lab 8} % The short title appears at the bottom of every slide, the full title is only on the title page

\author{Kasper Høj Lorenzen} % Your name
\institute[SDU Robotics] % Your institution as it will appear on the bottom of every slide, may be shorthand to save space
{
University of Southern Denmark \\ % Your institution for the title page
\medskip
\textit{kalor@mmmi.sdu.dk} % Your email address
}
\date{November 10, 2022} % Date, can be changed to a custom date

\begin{document}

\begin{frame}
\titlepage % Print the title page as the first slide
\end{frame}

\begin{frame}
\frametitle{Overview} % Table of contents slide, comment this block out to remove it
\tableofcontents % Throughout your presentation, if you choose to use \section{} and \subsection{} commands, these will automatically be printed on this slide as an overview of your presentation
\end{frame}

%----------------------------------------------------------------------------------------
%	PRESENTATION SLIDES
%----------------------------------------------------------------------------------------

% ------------------------------------------------
\section{Programming exercise 8}
% ------------------------------------------------

\begin{frame}
 \frametitle{Programming exercise 8 - Task Constrained Motion Planning}
	\begin{center}
        \includegraphics[width=\textwidth]{./rrt_constrain}
      \end{center}
\end{frame}



\begin{frame}
 \frametitle{Programming exercise 8 - Task Constrained Motion Planning}
  \begin{columns}
    \begin{column}{1\textwidth}
      \begin{itemize}
      \item Tips for programming exercise 8:
       \begin{itemize}
       \item We have constrained the movement of the bottle in the orientation (RPY)
       \item Find the right hyperparameters to enable the algorithm to solve the problem
       \item Change the goal frame "BottleGoal" in Scene.wc.xml to change the difficulty
       \end{itemize}
      \end{itemize}
    \end{column}
  \end{columns}
\end{frame}


\begin{frame}
 \frametitle{Programming exercise 8 - Task Constrained Motion Planning}
  \begin{columns}
    \begin{column}{1\textwidth}
      \begin{itemize}
      \item Task:
       \begin{itemize}
       \item Play around with the hyperparameters (step size, max error, random displacement, etc.)
       \end{itemize}
       \item Optional:
       \begin{itemize}
       \item Implement RRT-connect in the main loop to boost the performance of the algorithm significantly
       \end{itemize}       
      \end{itemize}
    \end{column}
  \end{columns}
\end{frame}

% \begin{frame}
%\frametitle{References}
%\footnotesize{
%\begin{thebibliography}{99} % Beamer does not support BibTeX so references must be inserted manually as below
%\bibitem[Ellekilde, Jorgensen, 2010]{robwork} L. P. Ellekilde and J. A. Jorgensen (2010)
%\newblock RobWork: A Flexible Toolbox for Robotics Research and Education
%\newblock \emph{ISR 2010 (41st International Symposium on Robotics) and ROBOTIK 2010 (6th German Conference on Robotics)}, 1 -- 7.
%\end{thebibliography}
%}
%\end{frame}

%----------------------------------------------------------------------------------------

\end{document} 
